% Options for packages loaded elsewhere
\PassOptionsToPackage{unicode}{hyperref}
\PassOptionsToPackage{hyphens}{url}
%
\documentclass[
  12pt,
  a4j]{ltjarticle}
\usepackage{amsmath,amssymb}
\usepackage{lmodern}
\usepackage{iftex}
\ifPDFTeX
  \usepackage[T1]{fontenc}
  \usepackage[utf8]{inputenc}
  \usepackage{textcomp} % provide euro and other symbols
\else % if luatex or xetex
  \usepackage{unicode-math}
  \defaultfontfeatures{Scale=MatchLowercase}
  \defaultfontfeatures[\rmfamily]{Ligatures=TeX,Scale=1}
\fi
% Use upquote if available, for straight quotes in verbatim environments
\IfFileExists{upquote.sty}{\usepackage{upquote}}{}
\IfFileExists{microtype.sty}{% use microtype if available
  \usepackage[]{microtype}
  \UseMicrotypeSet[protrusion]{basicmath} % disable protrusion for tt fonts
}{}
\makeatletter
\@ifundefined{KOMAClassName}{% if non-KOMA class
  \IfFileExists{parskip.sty}{%
    \usepackage{parskip}
  }{% else
    \setlength{\parindent}{0pt}
    \setlength{\parskip}{6pt plus 2pt minus 1pt}}
}{% if KOMA class
  \KOMAoptions{parskip=half}}
\makeatother
\usepackage{xcolor}
\IfFileExists{xurl.sty}{\usepackage{xurl}}{} % add URL line breaks if available
\IfFileExists{bookmark.sty}{\usepackage{bookmark}}{\usepackage{hyperref}}
\hypersetup{
  pdftitle={LaTeXなるべく書かないマン},
  pdfauthor={Yoshisaur},
  hidelinks,
  pdfcreator={LaTeX via pandoc}}
\urlstyle{same} % disable monospaced font for URLs
\usepackage[margin=1in]{geometry}
\usepackage{listings}
\newcommand{\passthrough}[1]{#1}
\lstset{defaultdialect=[5.3]Lua}
\lstset{defaultdialect=[x86masm]Assembler}
\usepackage{longtable,booktabs,array}
\usepackage{calc} % for calculating minipage widths
% Correct order of tables after \paragraph or \subparagraph
\usepackage{etoolbox}
\makeatletter
\patchcmd\longtable{\par}{\if@noskipsec\mbox{}\fi\par}{}{}
\makeatother
% Allow footnotes in longtable head/foot
\IfFileExists{footnotehyper.sty}{\usepackage{footnotehyper}}{\usepackage{footnote}}
\makesavenoteenv{longtable}
\usepackage{graphicx}
\makeatletter
\def\maxwidth{\ifdim\Gin@nat@width>\linewidth\linewidth\else\Gin@nat@width\fi}
\def\maxheight{\ifdim\Gin@nat@height>\textheight\textheight\else\Gin@nat@height\fi}
\makeatother
% Scale images if necessary, so that they will not overflow the page
% margins by default, and it is still possible to overwrite the defaults
% using explicit options in \includegraphics[width, height, ...]{}
\setkeys{Gin}{width=\maxwidth,height=\maxheight,keepaspectratio}
% Set default figure placement to htbp
\makeatletter
\def\fps@figure{htbp}
\makeatother
\setlength{\emergencystretch}{3em} % prevent overfull lines
\providecommand{\tightlist}{%
  \setlength{\itemsep}{0pt}\setlength{\parskip}{0pt}}
\setcounter{secnumdepth}{-\maxdimen} % remove section numbering
% adjust the position of an image to be displayed
\usepackage{float}
\let\origfigure\figure
\let\endorigfigure\endfigure
\renewenvironment{figure}[1][2] {
    \expandafter\origfigure\expandafter[H]
} {
    \endorigfigure
}

% add frame to a coding block
\usepackage{listings}
\lstset{
    numbers=left,
    frame=single,
    tabsize=2,
    breaklines=true,
}

% centre the title
\usepackage{titling}
\renewcommand{\maketitlehooka}{\null\mbox{}\vfill}
\renewcommand{\maketitlehookd}{\vfill\null}
\ifLuaTeX
  \usepackage{selnolig}  % disable illegal ligatures
\fi

\title{LaTeXなるべく書かないマン}
\author{Yoshisaur}
\date{20xx年 xx月 xx日}

\begin{document}
\maketitle

\clearpage

\hypertarget{ux7b2c1ux7ae0}{%
\section{第1章}\label{ux7b2c1ux7ae0}}

\hypertarget{yoshisaurux304aux524dux306fux4f55ux304cux3057ux305fux3044ux3093ux3060}{%
\subsection{Yoshisaur、お前は何がしたいんだ}\label{yoshisaurux304aux524dux306fux4f55ux304cux3057ux305fux3044ux3093ux3060}}

全部LaTeXではなくて、簡単なところはmarkdownでレポート書きたい

例えば箇条書きのLaTexは以下のようになっている

\begin{lstlisting}[title=LaTeXの箇条書き]
\begin{itemize}
\item hoge1
\item hoge2
\item hoge3
\end{itemize}
\end{lstlisting}

一方で、markdownは以下のようになっている

\begin{lstlisting}[title=markdownの箇条書き]
- hoge1
- hoge2
- hpge3
\end{lstlisting}

実際に上記のmarkdownで書かれた箇条書きは以下のように見える

\begin{itemize}
\tightlist
\item
  hoge1
\item
  hoge2
\item
  hpge3
\end{itemize}

明らかにLaTeXの記法よりmarkdownの記法の方が簡単で早い

また、markdownではサイズを調節して画像を貼ることも可能である

\begin{figure}
\centering
\includegraphics[width=0.75\textwidth,height=\textheight]{/data/images/sample.png}
\caption{画像のサンプル}
\end{figure}

また、テーブルも作成できる

\begin{longtable}[]{@{}ll@{}}
\toprule
用語 & 意味 \\
\midrule
\endhead
インジェラ & エチオピアの料理の1つ \\
テフ & イネ科の植物 \\
オフチョベット & 粉末状 \\
マブガッド & 水と混ぜる \\
リット & オフチョベットしてマブガッドしたテフ \\
\bottomrule
\end{longtable}

\hypertarget{ux7b2c2ux7ae0}{%
\section{第2章}\label{ux7b2c2ux7ae0}}

\hypertarget{markdownux306fux3044ux3044ux3068ux3053ux3060ux3089ux3051}{%
\subsection{markdownはいいとこだらけ?}\label{markdownux306fux3044ux3044ux3068ux3053ux3060ux3089ux3051}}

実はそうでもない、簡易的な記法なので表現力がない

例えば、markdownは常に左側にalignされるが、真ん中や右側にalignできない

ここは仕方ないので\passthrough{\lstinline!\\begin\{center\}!}(\passthrough{\lstinline!\\end\{center\}!})や\passthrough{\lstinline!\\begin\{flushright\}!}(\passthrough{\lstinline!\\end\{flushright\}!})を使う

こんな風にかく

\begin{lstlisting}[title=LaTeXの真ん中]
\begin{center}
真ん中
\end{center}
\end{lstlisting}

こうすると、以下のようになる

\begin{center}
真ん中
\end{center}

右側にalignする場合は

\begin{lstlisting}[title=LaTeXの右側]
\begin{flushright}
右側
\end{flushright}
\end{lstlisting}

こうすると、以下のようになる

\begin{flushright}
右側
\end{flushright}

また、数式もLaTeXの表現力に頼ることになる

\begin{lstlisting}[title=LaTeXの数式]
\begin{math}
\int^{b}_{a} f(x) dx = \lim_{n \to \infty} \sum^{n-1}_{i=0} f(x_{i}) \Delta x
\end{math}
\end{lstlisting}

と書いて

\begin{math}
\int^{b}_{a} f(x) dx = \lim_{n \to \infty} \sum^{n-1}_{i=0} f(x_{i}) \Delta x
\end{math}

となる

\begin{lstlisting}[title=LaTeXの数式2]
\begin{equation}
\int^{b}_{a} f(x) dx = \lim_{n \to \infty} \sum^{n-1}_{i=0} f(x_{i}) \Delta x
\end{equation}
\end{lstlisting}

と書いて

\begin{equation}
\int^{b}_{a} f(x) dx = \lim_{n \to \infty} \sum^{n-1}_{i=0} f(x_{i}) \Delta x
\end{equation}

こんな風にもなる

\hypertarget{ux7b2c3ux7ae0}{%
\section{第3章}\label{ux7b2c3ux7ae0}}

\hypertarget{ux3069ux3046ux3059ux308cux3070ux3044ux3044}{%
\subsection{どうすればいい}\label{ux3069ux3046ux3059ux308cux3070ux3044ux3044}}

markdownでできることはなるべくmarkdownで完結させて、LaTeXにしかできないことはLaTeXに頼ろう

\end{document}
